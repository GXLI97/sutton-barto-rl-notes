\usepackage{amsmath,amssymb,amsthm}
\usepackage{xcolor}
\usepackage{mathtools}
\usepackage{enumitem}
\usepackage{bbm}
\usepackage{semantic}
\usepackage{listings}
\usepackage{courier}
\usepackage{tikz}
\usetikzlibrary{calc}
\usepackage{color}
\usepackage{fancyhdr}
\usepackage{lastpage}
\usepackage{dsfont}
% \usepackage{subcaption}

\definecolor{codegreen}{rgb}{0,0.6,0}
\definecolor{codegray}{rgb}{0.5,0.5,0.5}
\definecolor{codepurple}{rgb}{0.58,0,0.82}
\definecolor{backcolour}{rgb}{0.95,0.95,0.92}
 
\lstdefinestyle{mystyle}{
    backgroundcolor=\color{backcolour},   
    commentstyle=\color{codegreen},
    keywordstyle=\color{magenta},
    numberstyle=\tiny\color{codegray},
    stringstyle=\color{codepurple},
    basicstyle=\footnotesize,
    breakatwhitespace=false,         
    breaklines=true,                 
    captionpos=b,                    
    keepspaces=true,                 
    numbers=left,                    
    numbersep=5pt,                  
    showspaces=false,                
    showstringspaces=false,
    showtabs=false,                  
    tabsize=2,
    basicstyle=\footnotesize\ttfamily
}

\lstset{style=mystyle}

\usepackage{hyperref}

\DeclareMathOperator*{\E}{\mathbb{E}}
\let\Pr\relax
\DeclareMathOperator*{\Pr}{\mathrm{Pr}}

\newcommand{\eps}{\epsilon}
\DeclareMathOperator*{\argmax}{arg\,max}
\DeclareMathOperator*{\argmin}{arg\,min}
\newcommand{\inprod}[1]{\langle #1 \rangle}
\newcommand{\innerprod}[1]{\left\langle #1 \right\rangle}
\newcommand{\R}{\mathbb{R}}

\newcommand{\norm}[1]{\left\lVert#1\right\rVert}
\newcommand{\abs}[1]{\left\lvert#1\right\rvert}
\newcommand{\paren}[1]{\left(#1\right)}

\newcommand{\sqb}[1]{\left[#1\right]}
\newcommand{\Mod}[1]{\ (\mathrm{mod}\ #1)}
\newcommand{\setmath}[1]{\left\{#1\right\}}
\newcommand*\xor{\mathbin{\oplus}}
\newcommand{\ceil}[1]{\left\lceil#1\right\rceil}
\newcommand{\floor}[1]{\left\lfloor#1\right\rfloor}
\newcommand{\bigoh}[1]{\mathcal{O}\paren{#1}}
\newcommand{\gaussian}[2]{\mathcal{N}\paren{#1, #2}}
\newcommand{\myfunc}[3]{#1\colon#2\to#3}
\newcommand{\poly}{\text{poly}}
\newcommand{\cond}[1]{\text{cond}\left(#1\right)}
\newcommand{\ind}[1]{\mathds{1}\{#1\}}
\newcommand{\rank}[1]{\text{rank}(#1)}
% \theoremstyle{definition}

\newenvironment{proofoutline}
 {\renewcommand\qedsymbol{}\proof[Proof outline]}
 {\endproof}

\newtheorem{theorem}{Theorem}[section]
\newtheorem*{lemma*}{Lemma}
\newtheorem*{theorem*}{Theorem}
\newtheorem{corollary}[theorem]{Corollary}
\newtheorem{remark}{Remark}
\newtheorem*{remark*}{Remark}
% \newtheorem{lemma}[theorem]{Lemma}
\newtheorem{lemma}{Lemma}[section]
% \newtheorem{example}[theorem]{Example}
\newtheorem{example}{Example}[section]
\newtheorem{observation}[theorem]{Observation}
\newtheorem{proposition}[theorem]{Proposition}
% \newtheorem{definition}[theorem]{Definition}
\newtheorem{definition}{Definition}[section]
\newtheorem{claim}[theorem]{Claim}
\newtheorem*{claim*}{Claim}
% \newtheorem{fact}[theorem]{Fact}
\newtheorem{fact}{Fact}[section]
\newtheorem{assumption}[theorem]{Assumption}
\newtheorem{conjecture}{Conjecture}[section]
\newtheorem*{conjecture*}{Conjecture}
% \newtheorem{problem}[theorem]{Problem}
\newtheorem{problem}{Problem}
% \numberwithin{problem}{subsection}
\newtheorem*{problem*}{Problem}
\newenvironment{solution}
  {\begin{proof}[Solution]}
  {\end{proof}}

\newcommand*\diff{\mathop{}\!\mathrm{d}}
\newcommand*\Diff[1]{\mathop{}\!\mathrm{d^#1}}

\DeclareMathOperator{\Tr}{Tr}
\DeclareMathOperator{\prox}{prox}
\DeclareMathOperator{\var}{var}
\DeclareMathOperator{\dom}{dom}
\DeclareMathOperator{\rk}{rk}
\DeclareMathOperator{\nul}{nul}
\DeclareMathOperator{\sign}{sign}
\DeclareMathOperator{\dual}{dual}
\DeclareMathOperator{\im}{im}
\DeclareMathOperator{\Span}{Span}
\DeclareMathOperator{\cost}{cost}
\DeclareMathOperator{\diam}{diam}

\newcommand{\exv}[1]{\E\sqb{#1}}
\newcommand{\expv}[1]{\mathrm{exp}\paren{#1}}
\newcommand{\prv}[1]{\Pr\sqb{#1}}
\newcommand{\logv}[1]{\log\paren{#1}}
\newcommand{\diag}[1]{\mathrm{diag}\paren{#1}}
\newcommand{\linf}{\ell_{\infty}}
\newcommand{\suchthat}{~\middle|~}

\makeatletter
\newcommand{\vo}{\vec{o}\@ifnextchar{^}{\,}{}}
\makeatother

\newcommand*{\vertbar}{\rule[-1ex]{0.5pt}{2.5ex}}
\newcommand*{\horzbar}{\rule[.5ex]{2.5ex}{0.5pt}}

\newcommand{\mvar}[1]{\mathit{#1}}
\newcommand{\qt}[1]{\mbox{`#1'}}

\usepackage{tcolorbox}
\usepackage{subfigure} 

% For citations
\usepackage{natbib}

% For algorithms
% \usepackage[section]{algorithm}
\usepackage{algorithm}
\usepackage{algorithmic}

% disable the below lines if you want auto-indent
% \newlength\tindent
% \setlength{\tindent}{\parindent}
% \setlength{\parindent}{0pt}
% \renewcommand{\indent}{\hspace*{\tindent}}

\numberwithin{equation}{section}
\numberwithin{algorithm}{problem}