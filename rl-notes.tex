\documentclass{article}
\usepackage[T1]{fontenc}
\usepackage[utf8]{inputenc}
 
\usepackage{tgpagella}
\usepackage{amsmath,amssymb,amsthm}
\usepackage{xcolor}
\usepackage{mathtools}
\usepackage{enumitem}
\usepackage{bbm}
\usepackage{semantic}
\usepackage{listings}
\usepackage{courier}
\usepackage{tikz}
\usetikzlibrary{calc}
\usepackage{color}
\usepackage{fancyhdr}
\usepackage{lastpage}
\usepackage{dsfont}
% \usepackage{subcaption}

\definecolor{codegreen}{rgb}{0,0.6,0}
\definecolor{codegray}{rgb}{0.5,0.5,0.5}
\definecolor{codepurple}{rgb}{0.58,0,0.82}
\definecolor{backcolour}{rgb}{0.95,0.95,0.92}
 
\lstdefinestyle{mystyle}{
    backgroundcolor=\color{backcolour},   
    commentstyle=\color{codegreen},
    keywordstyle=\color{magenta},
    numberstyle=\tiny\color{codegray},
    stringstyle=\color{codepurple},
    basicstyle=\footnotesize,
    breakatwhitespace=false,         
    breaklines=true,                 
    captionpos=b,                    
    keepspaces=true,                 
    numbers=left,                    
    numbersep=5pt,                  
    showspaces=false,                
    showstringspaces=false,
    showtabs=false,                  
    tabsize=2,
    basicstyle=\footnotesize\ttfamily
}

\lstset{style=mystyle}

\usepackage{hyperref}

\DeclareMathOperator*{\E}{\mathbb{E}}
\let\Pr\relax
\DeclareMathOperator*{\Pr}{\mathrm{Pr}}

\newcommand{\eps}{\epsilon}
\DeclareMathOperator*{\argmax}{arg\,max}
\DeclareMathOperator*{\argmin}{arg\,min}
\newcommand{\inprod}[1]{\langle #1 \rangle}
\newcommand{\innerprod}[1]{\left\langle #1 \right\rangle}
\newcommand{\R}{\mathbb{R}}

\newcommand{\norm}[1]{\left\lVert#1\right\rVert}
\newcommand{\abs}[1]{\left\lvert#1\right\rvert}
\newcommand{\paren}[1]{\left(#1\right)}

\newcommand{\sqb}[1]{\left[#1\right]}
\newcommand{\Mod}[1]{\ (\mathrm{mod}\ #1)}
\newcommand{\setmath}[1]{\left\{#1\right\}}
\newcommand*\xor{\mathbin{\oplus}}
\newcommand{\ceil}[1]{\left\lceil#1\right\rceil}
\newcommand{\floor}[1]{\left\lfloor#1\right\rfloor}
\newcommand{\bigoh}[1]{\mathcal{O}\paren{#1}}
\newcommand{\gaussian}[2]{\mathcal{N}\paren{#1, #2}}
\newcommand{\myfunc}[3]{#1\colon#2\to#3}
\newcommand{\poly}{\text{poly}}
\newcommand{\cond}[1]{\text{cond}\left(#1\right)}
\newcommand{\ind}[1]{\mathds{1}\{#1\}}
\newcommand{\rank}[1]{\text{rank}(#1)}
% \theoremstyle{definition}

\newenvironment{proofoutline}
 {\renewcommand\qedsymbol{}\proof[Proof outline]}
 {\endproof}

\newtheorem{theorem}{Theorem}[section]
\newtheorem*{lemma*}{Lemma}
\newtheorem*{theorem*}{Theorem}
\newtheorem{corollary}[theorem]{Corollary}
\newtheorem{remark}{Remark}
\newtheorem*{remark*}{Remark}
% \newtheorem{lemma}[theorem]{Lemma}
\newtheorem{lemma}{Lemma}[section]
% \newtheorem{example}[theorem]{Example}
\newtheorem{example}{Example}[section]
\newtheorem{observation}[theorem]{Observation}
\newtheorem{proposition}[theorem]{Proposition}
% \newtheorem{definition}[theorem]{Definition}
\newtheorem{definition}{Definition}[section]
\newtheorem{claim}[theorem]{Claim}
\newtheorem*{claim*}{Claim}
% \newtheorem{fact}[theorem]{Fact}
\newtheorem{fact}{Fact}[section]
\newtheorem{assumption}[theorem]{Assumption}
\newtheorem{conjecture}{Conjecture}[section]
\newtheorem*{conjecture*}{Conjecture}
% \newtheorem{problem}[theorem]{Problem}
\newtheorem{problem}{Problem}
% \numberwithin{problem}{subsection}
\newtheorem*{problem*}{Problem}
\newenvironment{solution}
  {\begin{proof}[Solution]}
  {\end{proof}}

\newcommand*\diff{\mathop{}\!\mathrm{d}}
\newcommand*\Diff[1]{\mathop{}\!\mathrm{d^#1}}

\DeclareMathOperator{\Tr}{Tr}
\DeclareMathOperator{\prox}{prox}
\DeclareMathOperator{\var}{var}
\DeclareMathOperator{\dom}{dom}
\DeclareMathOperator{\rk}{rk}
\DeclareMathOperator{\nul}{nul}
\DeclareMathOperator{\sign}{sign}
\DeclareMathOperator{\dual}{dual}
\DeclareMathOperator{\im}{im}
\DeclareMathOperator{\Span}{Span}
\DeclareMathOperator{\cost}{cost}
\DeclareMathOperator{\diam}{diam}

\newcommand{\exv}[1]{\E\sqb{#1}}
\newcommand{\expv}[1]{\mathrm{exp}\paren{#1}}
\newcommand{\prv}[1]{\Pr\sqb{#1}}
\newcommand{\logv}[1]{\log\paren{#1}}
\newcommand{\diag}[1]{\mathrm{diag}\paren{#1}}
\newcommand{\linf}{\ell_{\infty}}
\newcommand{\suchthat}{~\middle|~}

\makeatletter
\newcommand{\vo}{\vec{o}\@ifnextchar{^}{\,}{}}
\makeatother

\newcommand*{\vertbar}{\rule[-1ex]{0.5pt}{2.5ex}}
\newcommand*{\horzbar}{\rule[.5ex]{2.5ex}{0.5pt}}

\newcommand{\mvar}[1]{\mathit{#1}}
\newcommand{\qt}[1]{\mbox{`#1'}}

\usepackage{tcolorbox}
\usepackage{subfigure} 

% For citations
\usepackage{natbib}

% For algorithms
% \usepackage[section]{algorithm}
\usepackage{algorithm}
\usepackage{algorithmic}

% disable the below lines if you want auto-indent
% \newlength\tindent
% \setlength{\tindent}{\parindent}
% \setlength{\parindent}{0pt}
% \renewcommand{\indent}{\hspace*{\tindent}}

\numberwithin{equation}{section}
\numberwithin{algorithm}{problem}
\usepackage[margin=1in]{geometry}
\usepackage{parskip}

\title{Notes on Reinforcement Learning}
\author{Gene Li\\ \url{gene@ttic.edu}}

\begin{document}
\maketitle
\begin{abstract}
    These notes more or less follow the standard presentation found in Sutton and Barto. At times I have rearranged or omitted material to make it clearer for myself to understand.
\end{abstract}

% because the material starts on chapter 2...
\setcounter{section}{1}

\section{Multi-Armed Bandits}
The multi-armed bandit (MAB) problem represents a simple first swipe at understanding reinforcement learning. The structure of MAB is simple in the sense that we do not need to worry about state transitions; each action that the player takes in a round proceeds independently.

First, we define the multi-armed bandit setup.

\begin{definition}[Multi-Armed Bandit]
    Consider the following learning problem. At each round $t=1,2,...,T$, the player selects one action among $k$ choices, denoted $A_t \in [k]$ (typically called \textbf{arms}). They receive a reward $r_t \sim R_t(A_t)$, where $R_t$ is some distribution which depends on the action $a_t$. 
    
    We define the \textbf{expected reward} of action $a$ as:
    \begin{equation}
        q_\star(a) := \E[R_t(A_t)|A_t = a].
    \end{equation}
\end{definition}

Roughly speaking, the goal of the player is to select the action(s) $A_t$ which give the greatest reward $R_t$. However, a priori, the player does not know the distributions $R_t(a)$, so they must trade-off between learning these distributions (called \textit{exploration}) and selecting the action which they believe gives the greatest reward (called \textit{exploitation}).

\subsection{Greedy and $\eps$-Greedy}

Suppose at time $t$ we have a method for estimating the expected reward for each action $a$, i.e. we know the values:
\begin{equation}
    Q_t(a) \approx q_\star(a) \text{ for all } a\in[k].
\end{equation}

A simple way to get values of $Q_t(a)$ is to set it to the empirical mean of all rewards we accrued whenever we selected action $a$:
\begin{equation}
    Q_t(a) := \frac{\sum_{i=1}^{t-1} R_i \ind{A_i = a}}{\sum_{i=1}^{t-1} \ind{A_i = a}}.
\end{equation}

How can we exploit this knowledge? 
\begin{definition}[greedy]
    Select $a_t := \argmax_a Q_t(a)$.
\end{definition}

\begin{definition}[$\eps$-greedy]
    Select the greedy action ($a_t := \argmax_a Q_t(a)$) with probability $1-\eps$ and select a non-greedy action with probability $\eps$ uniformly at random.
\end{definition}

\subsection{Incremental Updates}
The main point is that we can write our updates of the action-value pairs $Q_t(a)$ in an incremental fashion, as:
\begin{equation}
    Q_{t+1}(a_t) \leftarrow Q_{t}(a_t) + \eta (R_t(a_t) - Q_{t}(a_t)),
\end{equation}

i.e. when we choose action $a_t$ we update our $Q_t$ proportional to the discrepancy between the true reward we got $R_t$ and our previous belief of the reward $R_t$.

Note that this generalizes the empirical mean estimation strategy, simply set $\eta = 1/n$. Generally, one can choose $\eta$ to be some function of $n$. If we choose $\eta = \bigoh(1)$, our $Q_t$ may never converge, but if our distributions $R_t$ are nonstationary, this may be desirable.

\subsection{Upper Confidence Bound}
\begin{definition}[Upper Confidence Bound]
Select actions as: $A_t := \argmax_a \{Q_t(a) + c \sqrt{\frac{\ln t}{N_t(a)}}\}$.
\end{definition}
There is some nice theory for why UCB is a good idea, but this book doesn't go into it. Intuitively this is trying to actions according to our upper confidence bound on the true mean. If we have less samples for an action, we will have a higher upper confidence bound so we should continue sampling those actions.

\subsection{Gradient Bandit}
Another thing we can do is model each action as a scalar value, $H_t(a) \in R$. Then we select each arm proportional to the soft-max over the $H_t$:
\begin{equation}
    \Pr[A_t = a] := \frac{e^{H_t(a)}}{\sum_b e^{H_t(b)}}.
\end{equation}
We can update the values $H_t$ according to a gradient ascent algorithm, as $H_{t+1} := H_t(a) + \eta \frac{\partial \E[R_t]}{\partial H_t(a)}$, which gives us the form in their Eq. 2.12.

\section{Finite Markov Decision Processes}

We begin by formalizing the problem definition of finite MDPs. In the MAB problem, we could pull each arm independently in the sense that the reward we earned did not depend on previous pulls. However, finite MDPs have this extra notion of ``state'' which complicates matters: our reward is a function of both the state and the action.

\begin{definition}[Finite MDP]
At each time step $t = 0,1,2,3...$, the player is in a state $S_t \in \mathcal{S}$. The player makes an action $A_t \in \mathcal{A}$, and receives reward $R_{t+1}(S_t, A_t) \in \mathcal{R}$, then transitioning to a new state $S_{t+1}$. For finite MDPs, $\mathcal{S}, \mathcal{A}, and \mathcal{R}$ are all finite sets.

We can completely characterize the MDP in terms of the \emph{dynamics}, which are given by $p(s', r|s,a) := \Pr [S_t = s', R_t = r | S_{t-1}=s, A_{t-1}=a]$ (This is where the Markov part comes in).

Note that using this, we can marginalize to compute other functions of interest.

\end{definition}

The name of the game for MDPs is for the player to maximize their expected cumulative reward of playing the MDP.

\paragraph{Time Horizons.} We need to have some notion of the ``time horizon''. The MDP can proceed forever, and we call this ``infinite time horizon''. Or, the MDP can terminate after some steps, and we restart the MDP. This is an example of episodic learning, where each trajectory is an episode of learning.

Note that when we are in the infinite time horizon, in order for the sequence $\sum_{k=t}^\infty R_k$ to converge, we can apply something called \emph{discounting}, which weights future rewards less: $\sum_{k=0}^\infty \gamma^k R_{t+k+1}$, for some $\gamma \in [0,1]$.

One point that is made is that we can essentially consider episodic learning within the infinite time horizon setting, by considering the terminating state as an infinite loop with 0 reward.

\subsection{Policy and Value Functions}
\begin{definition}
Define a \emph{policy} $\pi: \mathcal{S} \times \mathcal{A} \to [0,1]$ which expresses the probability of an agent taking action $a$ given that they are in state s, i.e. $\pi(a|s) = \Pr[A_t = a |S_t = s]$. 

Here, we consider \emph{static} policies, i.e. $\pi$ doesn't depend on $t$, however one can also consider changing policies $\pi_t$.
\end{definition}

\begin{definition}
Define the \emph{state-value function} $v_\pi(s): \mathcal{S} \to \R$ as the expected return when starting in $s$, following $\pi$:
\begin{equation}
    v_\pi(s) := \E_{\pi}\left[\sum_{k=0}^\infty \gamma^k R_{t+k+1} | S_t = s\right]
\end{equation}
\end{definition}

\begin{definition}
Define the \emph{action-value} function $q_\pi(s,a): \mathcal{S}\times \mathcal{A} \to \R$ as the expected return starting in state $s$, choosing $a$ initially and then following $\pi$:
\begin{equation}
    q_\pi(s,a) := \E_\pi \left[\sum_{k=0}^\infty \gamma^k R_{t+k+1} | S_t = s, A_t=a\right].
\end{equation}
\end{definition}

\subsection{Bellman Equations.}
Note the following condition for $v_\pi(s)$:
\begin{align}
    v_\pi(s) &:= \E_{\pi}\left[\sum_{k=0}^\infty \gamma^k R_{t+k+1} | S_t = s\right]\\
    &= \E_{\pi}\left[R_t + \gamma \sum_{k=0}^\infty \gamma^k R_{t+k+2} | S_t = s\right] \\
    &= \sum_a \pi(a|s) \sum_{s',r} p(s',r|s,a) [r+\gamma v_\pi(s')].\label{eq:be1}
\end{align}

The value function $v_\pi(s)$ satisfies a recursive equation, called the \emph{Bellman Equation}, and holds \emph{for any policy $\pi$}. $v_\pi$ is actually the unique solution to its Bellman equation.

\begin{definition}[Partial Ordering of Policies]
We define a partial ordering: $\pi \ge \pi'$ iff $v_\pi(s) \ge v_{\pi'}(s)$ for all $s\in \mathcal{S}$.
\end{definition}

\begin{claim}[Optimal Policy]
There exists a optimal policy $\pi^\star$ which is better than all other policies under the partial ordering ($\pi^\star \ge \pi$), and satisfies:
\begin{align}
    v_\star(s) &= \max_\pi v_\pi(s). \\
    q_\star(s,a) &= \max_\pi q_\pi(s,a).
\end{align}

Moreover, we have:
\begin{equation}
    q_\star(s,a) = \E[R_{t+1} + \gamma v_\star(S_{t+1})|S_t =s, A_t = a].
\end{equation}
\end{claim}

The policy $v_\star$ satisfies the Bellman Equations, but note in particular that the value of the state must be the expected return for the best action in that state:
\begin{align}
    v_\star(s) = \max_a q_\star(s,a) &= \max_a \E[R_{t+1} + \gamma v_\star(S_{t+1})|S_t=s, A_t=a] \\
    &= \max_a \sum_{s',r} p(s',r|s,a) [r+\gamma v_\star(s')] \label{eq:be2}
\end{align}
(Compare this to Eq. \ref{eq:be1}. Basically we are saying that our optimal policy is greedy w.r.t. to the maximizers, i.e. the probability mass $\sum_{a \in \{a^\star\}} \pi(a|s) = 1$ for the optimal action $\{a^\star\}= \argmax_a q_\star(s,a)$ - see the discussion on pg. 64)

In addition, we can also write Bellman Equations for $q_\star$:
\begin{align}
    q_\star(s,a) &= \E[R_{t+1} + \gamma \max_{a'} q_\star(S_{t+1},a')|S_t=s, A_t=a]\\
    &= \sum_{s',r} p(s',r|s,a) [r+\gamma \max_{a'} q_\star(s',a')].
\end{align}

The key point is that if we know the system dynamics perfectly, we can solve the Bellman Equations to obtain the optimal player strategy. However (1) often we do not know the system dynamics, (2) solving the Bellman Equations naively by exhaustive search is a bad option computationally. So we need to come up with methods which approximately solve the Bellman Equations.

\section{Dynamic Programming}
Here, we consider a way to solve the Bellman Equation (approximately) if we assume perfect knowledge of the state dynamics (i.e. we know $p(s',r|s,a)$). Our objective is to compute the state-value $v_\pi(s)$ and action-value functions $q_\pi(s,a)$ given knowledge of $p$.

\subsection{Policy Evaluation: Computing $v_\pi$}

Suppose we have some policy $\pi$. Our task is to compute the value function $v_\pi(s)$ for all $s\in \mathcal{S}$. From our Bellman Equation (\ref{eq:be1}):
\begin{equation}
    v_\pi(s) = \sum_a \pi(a|s) \sum_{s',r} p(s',r|s,a) [r+\gamma v_\pi(s')].
\end{equation}

\begin{claim}[Iterative Policy Evaluation.]
$v_\pi(s)$ is a fixed point of this recursive Bellman equation, so we can actually solve for $v_\pi$ by initializing $v_0$ and updating until convergence:

\begin{equation}
    v_{k+1}(s) \leftarrow \sum_a \pi(a|s) \sum_{s',r} p(s',r|s,a) [r+\gamma v_k(s')]. \label{eq:ipe}
\end{equation}

At each step, we update the value of $v_{k+1}$ for each $s$, then use those new updates on the RHS of the equation.

The claim is that this converges eventually. (How fast is another matter that is not addressed by Sutton-Barto.)
\end{claim}

\subsection{Policy Improvement: How to change $\pi$}
\begin{theorem}[Policy Improvement Theorem]
Let $\pi$, $\pi'$ be deterministic policies such that for all $s\in\mathcal{S}$:
\begin{equation}
    q_\pi(s,\pi'(s)) \ge v_\pi(s).
\end{equation}

In other words, at each state $s$, following $\pi'$ for one step, then switching back to $\pi$, is always at least as good as following $\pi$. Then we must have $\pi'\ge \pi$, i.e. for all states $s\in\mathcal{S}$
\begin{equation}
    v_{\pi'}(s) \ge v_\pi(s).
\end{equation}
This means that following $\pi'$ always is at least as good as following $\pi$ always.
\end{theorem}

\begin{proof}
Proof deferred to Sutton-Barto pg. 78.
\end{proof}

This gives us a way to update policies. Suppose we have computed some $q_\pi(s,a)$ for a given policy $\pi$. For instance, $q_\pi(s,a)$ can be computed from $v_\pi(s)$. Now consider the new policy:
\begin{equation}
    \pi'(s) := \argmax_a q_\pi(s,a).\label{eq:pimp}
\end{equation}

This new policy satisfies the conditions of the Policy Improvement Theorem, so we see that $\pi'$ is at least as good as $\pi$.

A second point is that if $\pi'$ as calculated above performs the same as $\pi$ ($v_\pi(s) = v_{\pi'}(s)$ for all $s$), then it must be the case that $\pi$ and $\pi'$ are both optimal policies. This can be seen by relating this to the Bellman Equation for optimality (\ref{eq:be2}).

Conversely, if $\pi'$ is not an optimal policy, then the update must give us a strict improvement.

The last note is that these results on deterministic policies can be extended to stochastic policies, but this is not shown in Sutton-Barto.

\subsection{Policy/Value Iteration}
Policy iteration just means applying both policy evaluation and policy improvement one after the other to eventually find $\pi^\star$, an optimal policy, so it's not particularly interesting.

However, note that evaluation and improvement are not the same in terms of computation. Evaluation requires us iteratively updating value functions until we reach convergence, and it is unclear how long this may take (typically we can cut it off when we do not make sufficient improvement). Improvement is much easier, because it is a one-step update (Eq. \ref{eq:pimp}). This leads us to try to make policy evaluation faster.

The easiest thing to do is only apply one step of the update (\ref{eq:ipe}). We can actually dress up the evaluation and improvement step into one equation.

\begin{definition}[Value Iteration Update]
Update the values as:
\begin{equation}
v_{k+1}(s) \leftarrow \max_a \sum_{s',r} p(s',r|s,a) [r+\gamma v_k(s')]. 
\end{equation}

At the end, the explicit policy that we compute is:
\begin{equation}
    \pi(s) = \argmax_a \sum_{s',r} p(s',r|s,a) [r+\gamma v_\text{end}(s')]
\end{equation}

\end{definition}

\section{Monte Carlo Methods}
We take one step further and consider when we do not have perfect knowledge of the system dynamics, i.e. our player must also estimate these system dynamics.

A note about this section. We go back to looking at the episodic setting for finite MDPs. Only in between episodes do we ever make any updates.

\subsection{MC Estimation: Learning $v_\pi$}

Essentially, we average the empirical values of $v_\pi$ that we get in order to obtain an estimate. Recall that $v_\pi(s)$ is the expected reward of the future, given that we start in state $s$. Therefore, once we see $s$, we can compute the empirical reward over the rest of the episode and include that in our estimates. 

\begin{definition}[First Visit MC]
We sample episodes under a policy $\pi$. For each state $s$, we average the aggregate reward for the episode using only first visits to $s$.
\end{definition}

\begin{definition}[Every Visit MC]
We sample episodes under a policy $\pi$. For each state $s$, we average the aggregate reward for the episode using all visits to $s$.
\end{definition}

As the number of visits to $s$ goes to infinity, both converge to $v_\pi(s)$ by concentration of average. The main reason why we consider first visit MC is because we have unbiased estimators, which makes analysis easier.

\subsection{MC Estimation: Learning $q_\pi(s,a)$}

If we do not have access to the system dynamics, state values $v_\pi$ are not sufficient. We need to actually estimate the state-action value $q_\pi(s,a)$ to determine an optimal policy. Note that this can be much harder - before we were estimating $|\mathcal{S}|$ values, but now we are be estimating $|\mathcal{S}|\times |\mathcal{A}|$ values.


\end{document}